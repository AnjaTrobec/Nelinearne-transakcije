% Options for packages loaded elsewhere
\PassOptionsToPackage{unicode}{hyperref}
\PassOptionsToPackage{hyphens}{url}
%
\documentclass[
]{article}
\title{GEN-I: Nelinearne transakcije}
\usepackage{etoolbox}
\makeatletter
\providecommand{\subtitle}[1]{% add subtitle to \maketitle
  \apptocmd{\@title}{\par {\large #1 \par}}{}{}
}
\makeatother
\subtitle{Projekt pri predmetu: Matematika z racunalnikom}
\author{Anja Trobec}
\date{Maj 2022}

\usepackage{amsmath,amssymb}
\usepackage{lmodern}
\usepackage{iftex}
\ifPDFTeX
  \usepackage[T1]{fontenc}
  \usepackage[utf8]{inputenc}
  \usepackage{textcomp} % provide euro and other symbols
\else % if luatex or xetex
  \usepackage{unicode-math}
  \defaultfontfeatures{Scale=MatchLowercase}
  \defaultfontfeatures[\rmfamily]{Ligatures=TeX,Scale=1}
\fi
% Use upquote if available, for straight quotes in verbatim environments
\IfFileExists{upquote.sty}{\usepackage{upquote}}{}
\IfFileExists{microtype.sty}{% use microtype if available
  \usepackage[]{microtype}
  \UseMicrotypeSet[protrusion]{basicmath} % disable protrusion for tt fonts
}{}
\makeatletter
\@ifundefined{KOMAClassName}{% if non-KOMA class
  \IfFileExists{parskip.sty}{%
    \usepackage{parskip}
  }{% else
    \setlength{\parindent}{0pt}
    \setlength{\parskip}{6pt plus 2pt minus 1pt}}
}{% if KOMA class
  \KOMAoptions{parskip=half}}
\makeatother
\usepackage{xcolor}
\IfFileExists{xurl.sty}{\usepackage{xurl}}{} % add URL line breaks if available
\IfFileExists{bookmark.sty}{\usepackage{bookmark}}{\usepackage{hyperref}}
\hypersetup{
  pdftitle={GEN-I: Nelinearne transakcije},
  pdfauthor={Anja Trobec},
  hidelinks,
  pdfcreator={LaTeX via pandoc}}
\urlstyle{same} % disable monospaced font for URLs
\usepackage[margin=1in]{geometry}
\usepackage{graphicx}
\makeatletter
\def\maxwidth{\ifdim\Gin@nat@width>\linewidth\linewidth\else\Gin@nat@width\fi}
\def\maxheight{\ifdim\Gin@nat@height>\textheight\textheight\else\Gin@nat@height\fi}
\makeatother
% Scale images if necessary, so that they will not overflow the page
% margins by default, and it is still possible to overwrite the defaults
% using explicit options in \includegraphics[width, height, ...]{}
\setkeys{Gin}{width=\maxwidth,height=\maxheight,keepaspectratio}
% Set default figure placement to htbp
\makeatletter
\def\fps@figure{htbp}
\makeatother
\setlength{\emergencystretch}{3em} % prevent overfull lines
\providecommand{\tightlist}{%
  \setlength{\itemsep}{0pt}\setlength{\parskip}{0pt}}
\setcounter{secnumdepth}{-\maxdimen} % remove section numbering
\ifLuaTeX
  \usepackage{selnolig}  % disable illegal ligatures
\fi

\begin{document}
\maketitle

{
\setcounter{tocdepth}{2}
\tableofcontents
}
NAVODILA ZA IZDELAVO PROJEKTA

Imamo mesecno nelinearno transakcijo za elektricno energijo, kjer se
lahko znotraj dolocenih omejitev za vsako uro znotraj meseca dobave
lastnik opcijskosti odloci, koliko el. energije bo prevzeli/dobavil.
Ovrednotimo jo napram mnozici cenovnih scenarijev, tako da za vsak
cenovni scenarij dobimo njen profit. Cenovni scenariji so mozne
prihodnje cene dobave, oblikovani tako, da ustrezno popisejo
verjetnostno porazdelitev prihodnje cene v smislu njene srednje
vrednosti in standardne deviacije (volatilnosit). Za to transakcijo
zelimo poiskati njen ekvivalent standardne Evropske opcije. Kaksni so
parametri ekvivalenta te opcije, kot kolicina, cena (strike), stran
(nakup/prodaja) in tip (call/put) opcije?

PROJEKT

\begin{enumerate}
\def\labelenumi{\arabic{enumi}.}
\tightlist
\item
  TEORETICNI UVOD
\end{enumerate}

Moja glavna naloga je poiskati Evropsko opcijo, ki se najbolje pribliza
transakciji predstavljeni v podatkih. Vhodni podatki so pari, v katerih
prva komponenta predstavlja ceno in druga profit pri dani ceni.

Evropski opciji, ki bo ekvivalentna transakciji, je potrebno dolociti
parametre. To so:

\begin{itemize}
\tightlist
\item
  v naprej dogovorjena izvrsilna cena (angl. strike price),
\item
  kolicina in
\item
  placana ali prejeta premija.
\end{itemize}

Dolociti moramo \textbf{tip opcije}. Lahko gre za:

\begin{itemize}
\tightlist
\item
  \textbf{nakupno opcijo} (\emph{angl. call option}) ali
\item
  \textbf{prodajno opcijo} (\emph{angl. put option}).
\end{itemize}

Znotraj obeh tipov opcij, locimo se pozicijo, ki jo zavzamemo. Lahko smo
v vlogi izdajatelja opcije (\emph{angl. option writer}) in v tem primeru
\textbf{opcijo prodamo} ali pa \textbf{opcijo kupimo} in s tem postanemo
lastnik opcije (\emph{angl. option buyer}).

S tem smo prisli do stirih razlicnih situacij, med katerimi iscemo
tisto, ki najbolje opise dano transakcijo. Podrobneje si poglejmo vsako
izmed moznih izbir.

\begin{enumerate}
\def\labelenumi{\arabic{enumi}.}
\tightlist
\item
  NAKUP EVROPSKE NAKUPNE OPCIJE
\end{enumerate}

Nakupna opcija podeljuje lastniku pravico za nakup dolocenega financnega
instrumenta (angl. \emph{underlying asset}) po vnaprej doloceni
izvrsilni ceni na dolocen dan (kadar govorimo o Evropski opciji) ali do
dolocenega dne (kadar imamo opravka z Amerisko opcijo). Lastniku nakupna
opcija ne predstavlja obveznosti, pac pa priloznost (recemo, da mu nudi
opcijskost), da opcijo izvrsi v primeru, ce cena financnega instrumenta
na trgu naraste. Za nakupno opcijo recemo, da je:

\begin{itemize}
\tightlist
\item
  \textbf{in the money} - ce je cena financnega instrumenta nad
  izvrsilno ceno,
\item
  \textbf{at the money} - ce sta cena financnega instrumenta in
  izvrsilna cena enaki,
\item
  \textbf{put of the money} - ce je cena financnega instumenta pod
  izvrsilno ceno.
\end{itemize}

Opazimo, da ima kupec evropske nakupne opcije neomejen dobicek in na
drugi strani izgubo omejeno s premijo. Drugace povedano, najvec kar
lahko kupec izgubi je premija, ki jo placa za nakup opcije v primeru, da
opcije ne izvrsi.

Formula za vrednotenje izplacil opcije ob casu t:
\[ V_t = max(S_t-K,0) = (S_t - K)^+ \]

\begin{enumerate}
\def\labelenumi{\arabic{enumi}.}
\setcounter{enumi}{1}
\tightlist
\item
  NAKUP EVROPSKE PRODAJNE OPCIJE
\end{enumerate}

Prodajna opcija podeljuje lastniku pravico za prodajo dolocenega
financnega instrumenta (angl. \emph{underlying asset}) po vnaprej
doloceni izvrsilni ceni na dolocen dan (kadar govorimo o Evropski
opciji) ali do dolocenega dne (kadar imamo opravka z Amerisko opcijo).
Lastniku nakupna opcija ne predstavlja obveznosti, pac pa priloznost
(recemo, da mu nudi opcijskost), da opcijo izvrsi v primeru, ce cena
financnega instrumenta na trgu pade. Za nakupno opcijo recemo, da je:

\begin{itemize}
\tightlist
\item
  \textbf{in the money} - ce je cena financnega instrumenta pod
  izvrsilno ceno,
\item
  \textbf{at the money} - ce sta cena financnega instrumenta in
  izvrsilna cena enaki,
\item
  \textbf{put of the money} - ce je cena financnega instumenta nad
  izvrsilno ceno.
\end{itemize}

Enako kot pri nakupu evropske nakupne opcije lahko opazimo, da ima kupec
evropske prodajne opcije neomejen dobicek in na drugi strani izgubo
omejeno s premijo. Drugace povedano, najvec kar lahko kupec izgubi je
premija, ki jo placa za nakup opcije v primeru, da opcije ne izvrsi.
Formula za vrednotenje izplacil opcije ob casu t:
\[ V_t = max(K-S_t,0) = (K-S_t)^+ \]

\begin{enumerate}
\def\labelenumi{\arabic{enumi}.}
\setcounter{enumi}{2}
\tightlist
\item
  PRODAJA EVROPSKE NAKUPNE OPCIJE
\end{enumerate}

Zdaj se postavimo v vlogo izdajatelja opcije. S tem ko opcijo prodamo,
se zavezemo k izplacilu v primeru, da kupec opcijo izvrsi. Torej je v
tem primeru dobicek navzgor omejen s prejeto premijo in izguba navzdol
neomejena v primeru, da cena na trgu naraste.

Formula za vrednotenje izplacil opcije ob casu t:
\[ V_t = min(K-S_t,0) - max(S_t-K,0) = (K-S_t)^- - (S_t - K)^+ \]

\begin{enumerate}
\def\labelenumi{\arabic{enumi}.}
\setcounter{enumi}{3}
\tightlist
\item
  PRODAJA EVROPSKE PRODAJNE OPCIJE
\end{enumerate}

Zadnji scenarij pa je prodaja evropske prodajne opcije. Izdajatelj
prodajne opcije opcijo proda, zanjo prejme premijo in se zaveze k
izplacilu v primeru, da lastnik opcijo izvrsi. Ponovno je njegova izguba
navzdol neomejena, medtem ko je dobicek navzgor omejen s prejeto
premijo.

Formula za vrednotenje izplacil opcije ob casu t:
\[ V_t = min(S_t-K,0) = (S_t-K)^-  \]

Poglejmo si izplacila vseh stirih opcij na spodnji sliki.

\begin{center}\includegraphics{projekt_files/figure-latex/unnamed-chunk-2-1} \end{center}

\begin{enumerate}
\def\labelenumi{\arabic{enumi}.}
\setcounter{enumi}{1}
\tightlist
\item
  PRISTOP K RESEVANJU PROBLEMA
\end{enumerate}

Resevanje problema se lotimo tako, da vhodne podatke preberemo in
graficno upodobimo. V nekaterih primerih bo ze iz zacetne slike jasno,
kateri izmed stirih situacij pripada transakcija. En izmed takih je
predstavljen v nadaljevanju.

\begin{center}\includegraphics{projekt_files/figure-latex/unnamed-chunk-4-1} \end{center}

Iz slike je jasno razvidno, da gre za nakup evropske prodajne opcije.
Torej vse kar nam preostane je le se dolocitev parametrov. Dolociti
moramo izvrsilno ceno, kolicino in premijo. Kako to najlazje storimo?

Ocitno je, da so vse stiri upodobitve opcij sestavljene iz dveh premic,
ki imata v vseh primerih zelo podobno oblika. Ena izmed premic je vselej
vzporedna x osi, druga pa ima pozitiven ali negativen naklon. Ideja je,
da vsako vhodno transakcijo aproksimiramo s kombinacijo teh dveh crt,
ugotovimo za katero opcijo gre in iz naklona in zacetne vrednosti
optimalnih premic dolocimo preostale parametre.

Premica, ki bo vselej vodoravna doloca premijo: \[ y =  premija \]
Premica s pozitivnim ali negativnim naklonom pa kolicino (Q):

\[ y =  kolicina * S_t + n\] Iz presecisca zgornjih premic dobimo
izvrsilno ceno (K):

\[ K = \frac{premija - n}{kolicina} \] Iz vsega opisanega lahko
sestavimo algoritem, ki bo izracunal iskane parametre in odgovoril na
vprasanje za katero opcijo gre. Algoritem sprejme tabelo sestavljeno iz
dveh stolpcev. V prvem stolpcu najdemo ceno financnega instrumenta in v
drugem stolpcu najdemo profit pri tej ceni. Algoritem podatke prebere in
odloci, za kaksno vrsto opcije gre. To stori na naslednji nacin:

\begin{enumerate}
\def\labelenumi{\arabic{enumi}.}
\item
  Izracuna korelacijo med podatki. ** Ce je korelacija negativna, takoj
  vemo, da gre za nakup prodajne opcije ali za prodajo nakupne opcije.
  ** Ce je korelacija pozitivna, pa nam preostane prodaja prodajne
  opcije ali nakup nakupne opcije.
\item
  Zdaj izbiramo le se med dvema moznostima. Odlocitev ali gre za nakupno
  ali prodajno opcijo sprejmemo na podlagi velikosti napake, ki se
  pojavi pri aproksimaciji z eno ali z drugo kombinacijo premic.
  Izbiramo torej med kombinacijama:
\end{enumerate}

\begin{itemize}
\tightlist
\item
  prva premica bo vodoravna in druga z nenicelnim naklonom ali
\item
  prva premica bo imela nenicelni naklon in druga bo vodoravna. Z
  enostavno primerjavo napak smo nasli ustrezno obliko za aproksimacijo.
\end{itemize}

Prisli smo do jedra algoritma v katerem iscemo optimalno prileganje
izbrane opcije na dane podatke. Potrebno bo iskati najboljse prileganje
obeh omenjenih premic. Premico z nenicelnim naklonom dobimo s pomocjo
linearne regresije, medtem ko vodoravno premico dolocimo preko
povprecnih profitov v ``vodoravnem delu podatkov''.

\begin{center}\includegraphics{projekt_files/figure-latex/unnamed-chunk-6-1} \end{center}

\end{document}
