% Options for packages loaded elsewhere
\PassOptionsToPackage{unicode}{hyperref}
\PassOptionsToPackage{hyphens}{url}
%
\documentclass[
]{article}
\title{GEN-I: Nelinearne transakcije}
\usepackage{etoolbox}
\makeatletter
\providecommand{\subtitle}[1]{% add subtitle to \maketitle
  \apptocmd{\@title}{\par {\large #1 \par}}{}{}
}
\makeatother
\subtitle{Poročilo}
\author{Anja Trobec}
\date{Maj 2022}

\usepackage{amsmath,amssymb}
\usepackage{lmodern}
\usepackage{iftex}
\ifPDFTeX
  \usepackage[T1]{fontenc}
  \usepackage[utf8]{inputenc}
  \usepackage{textcomp} % provide euro and other symbols
\else % if luatex or xetex
  \usepackage{unicode-math}
  \defaultfontfeatures{Scale=MatchLowercase}
  \defaultfontfeatures[\rmfamily]{Ligatures=TeX,Scale=1}
\fi
% Use upquote if available, for straight quotes in verbatim environments
\IfFileExists{upquote.sty}{\usepackage{upquote}}{}
\IfFileExists{microtype.sty}{% use microtype if available
  \usepackage[]{microtype}
  \UseMicrotypeSet[protrusion]{basicmath} % disable protrusion for tt fonts
}{}
\makeatletter
\@ifundefined{KOMAClassName}{% if non-KOMA class
  \IfFileExists{parskip.sty}{%
    \usepackage{parskip}
  }{% else
    \setlength{\parindent}{0pt}
    \setlength{\parskip}{6pt plus 2pt minus 1pt}}
}{% if KOMA class
  \KOMAoptions{parskip=half}}
\makeatother
\usepackage{xcolor}
\IfFileExists{xurl.sty}{\usepackage{xurl}}{} % add URL line breaks if available
\IfFileExists{bookmark.sty}{\usepackage{bookmark}}{\usepackage{hyperref}}
\hypersetup{
  pdftitle={GEN-I: Nelinearne transakcije},
  pdfauthor={Anja Trobec},
  hidelinks,
  pdfcreator={LaTeX via pandoc}}
\urlstyle{same} % disable monospaced font for URLs
\usepackage[margin=1in]{geometry}
\usepackage{graphicx}
\makeatletter
\def\maxwidth{\ifdim\Gin@nat@width>\linewidth\linewidth\else\Gin@nat@width\fi}
\def\maxheight{\ifdim\Gin@nat@height>\textheight\textheight\else\Gin@nat@height\fi}
\makeatother
% Scale images if necessary, so that they will not overflow the page
% margins by default, and it is still possible to overwrite the defaults
% using explicit options in \includegraphics[width, height, ...]{}
\setkeys{Gin}{width=\maxwidth,height=\maxheight,keepaspectratio}
% Set default figure placement to htbp
\makeatletter
\def\fps@figure{htbp}
\makeatother
\setlength{\emergencystretch}{3em} % prevent overfull lines
\providecommand{\tightlist}{%
  \setlength{\itemsep}{0pt}\setlength{\parskip}{0pt}}
\setcounter{secnumdepth}{-\maxdimen} % remove section numbering
\ifLuaTeX
  \usepackage{selnolig}  % disable illegal ligatures
\fi

\begin{document}
\maketitle

NAVODILA ZA IZDELAVO PROJEKTA

Imamo mesečno nelinearno transakcijo za električno energijo, kjer se
lahko znotraj določenih omejitev za vsako uro znotraj meseca dobave
lastnik opcijskosti odloči, koliko el. energije bo prevzeli/dobavil.
Ovrednotimo jo napram množici cenovnih scenarijev, tako da za vsak
cenovni scenarij dobimo njen profit. Cenovni scenariji so možne
prihodnje cene dobave, oblikovani tako, da ustrezno popišejo
verjetnostno porazdelitev prihodnje cene v smislu njene srednje
vrednosti in standardne deviacije (volatilnosit). Za to transakcijo
želimo poiskati njen ekvivalent standardne Evropske opcije. Kakšni so
parametri ekvivalenta te opcije, kot količina, cena (strike), stran
(nakup/prodaja) in tip (call/put) opcije?

PROJEKT

\begin{enumerate}
\def\labelenumi{\arabic{enumi}.}
\tightlist
\item
  TEORETIČNI UVOD
\end{enumerate}

Moja naloga je poiskati Evropsko opcijo, ki najboljše opiše nelinearno
transakcijo predstavljeno v podatkih. Vhodni podatki so pari, v katerih
prva komponenta predstavlja ceno (\emph{\(S\)} - \emph{price}) in druga
komponenta izplačilo pri dani ceni (\emph{\(V\)} - \emph{profit}). Cena
je izrazena v EUR/MWh, izplačilo pa v EUR.

\[(S,V) \]

Iskani Evropski opciji, ki bo ekvivalentna transakciji, je potrebno
določiti najslednje parametre:

\begin{itemize}
\tightlist
\item
  izvršilno cena (\(K\), \emph{angl. strike price}),
\item
  količino (\(Q\)) in
\item
  plačano ali prejeto premijo (\emph{angl. option premium}).
\end{itemize}

Določiti je potrebno tudi \textbf{tip opcije}. Lahko gre za:

\begin{itemize}
\tightlist
\item
  \textbf{call opcijo} (\emph{angl. call option}) ali
\item
  \textbf{put opcijo} (\emph{angl. put option}).
\end{itemize}

Znotraj obeh tipov opcij, ločimo še dve poziciji, kateri lahko
zavzamemo. Lahko smo v vlogi izdajatelja opcije (\emph{angl. option
writer}) in v tem primeru \textbf{opcijo prodamo} ali pa \textbf{opcijo
kupimo} in s tem postanemo lastnik opcije (\emph{angl. option buyer}).

To nas pripelje do štirih različnih situacij, med katerimi iščemo tisto,
ki najbolje opiše dano transakcijo. Podrobneje si oglejmo vsako izmed
možnih izbir.

\begin{enumerate}
\def\labelenumi{\arabic{enumi}.}
\tightlist
\item
  NAKUP EVROPSKE CALL OPCIJE
\end{enumerate}

\textbf{Call opcija} podeljuje \textbf{lastniku (kupcu opcije)} pravico
za nakup določenega inštrumenta (\emph{angl. underlying asset}) po
vnaprej določeni izvršilni ceni na dolocen dan (kadar govorimo o
Evropski opciji) ali do določenega dne (kadar imamo opravka z Ameriško
opcijo). Lastniku call opcija ne predstavlja obveznosti, pač pa
priložnost (rečemo, da mu nudi opcijskost), da opcijo izvrši v primeru,
če cena inštrumenta na trgu naraste. Za call opcijo rečemo, da je:

\begin{itemize}
\tightlist
\item
  \textbf{in the money} - kadar je cena inštrumenta nad izvršilno ceno,
\item
  \textbf{at the money} - kadar sta cena inštrumenta in izvršilna cena
  enaki,
\item
  \textbf{put of the money} - kadar je cena instumenta pod izvršilno
  ceno.
\end{itemize}

Opazimo, da ima kupec evropske call opcije \textbf{neomejen dobiček} in
na drugi strani \textbf{izgubo omejeno s plačano premijo}. Drugače
povedano, največ kar lahko kupec izgubi je premija, ki jo plača za nakup
opcije v primeru, da opcije ne izvrši.

Formula za vrednotenje izplačil opcije ob času t:
\[ V_t = max(S_t-K,0) - premija = (S_t - K)^+ - premija\]

\begin{enumerate}
\def\labelenumi{\arabic{enumi}.}
\setcounter{enumi}{1}
\tightlist
\item
  NAKUP EVROPSKE PUT OPCIJE
\end{enumerate}

\textbf{Put opcija} podeljuje \textbf{lastniku (kupcu opcije) } pravico
za prodajo določenega inštrumenta po vnaprej določeni izvršilni ceni na
določen dan ali do določenega dne. Lastniku call opcija ne predstavlja
obveznosti, pač pa priložnost (rečemo, da mu nudi opcijskost), da opcijo
izvrši v primeru, če cena inštrumenta na trgu pade. Za call opcijo
rečemo, da je:

\begin{itemize}
\tightlist
\item
  \textbf{in the money} - kadar je cena inštrumenta pod izvršilno ceno,
\item
  \textbf{at the money} - kadar sta cena inštrumenta in izvršilna cena
  enaki,
\item
  \textbf{put of the money} - kadar je cena instumenta nad izvršilno
  ceno.
\end{itemize}

Enako kot pri nakupu evropske call opcije lahko opazimo, da ima kupec
evropske put opcije \textbf{neomejen dobicek} in na drugi strani
\textbf{izgubo omejeno s plačano premijo}. Drugače povedano, največ kar
lahko kupec izgubi je premija, ki jo plača za nakup opcije v primeru, da
opcije ne izvrši. Formula za vrednotenje izplacil opcije ob času t:
\[ V_t = max(K-S_t,0) - premija = (K-S_t)^+ - premija\]

\begin{enumerate}
\def\labelenumi{\arabic{enumi}.}
\setcounter{enumi}{2}
\tightlist
\item
  PRODAJA EVROPSKE CALL OPCIJE
\end{enumerate}

Zdaj se postavimo v vlogo izdajatelja opcije. S tem ko \textbf{opcijo
prodamo}, zanjo \textbf{prejmemo premijo} in se zavežemo k izplačilu v
primeru, da kupec opcijo ob dospelosti izvrši. Torej je v primeru
prodaje call opcije \textbf{dobicek navzgor omejen s prejeto premijo} in
\textbf{izguba navzdol neomejena} (do velike izgube pride, če cena
inštrumenta na trgu naraste).

Formula za vrednotenje izplačil opcije ob času t:
\[ V_t = premija - max(S_t-K,0) = premija - (S_t - K)^+ \]

\begin{enumerate}
\def\labelenumi{\arabic{enumi}.}
\setcounter{enumi}{3}
\tightlist
\item
  PRODAJA EVROPSKE PUT OPCIJE
\end{enumerate}

Zadnji scenarij pa je prodaja evropske put opcije. Kot izdajatelj put
opcije, \textbf{opcijo prodamo}, zanjo \textbf{prejmemo premijo} in se
zavežemo k izplačilu v primeru, da lastnik opcijo ob dospelosti izvrši.
Ponovno je \textbf{izguba navzdol neomejena}, medtem ko je
\textbf{dobiček navzgor omejen s prejeto premijo}.

Formula za vrednotenje izplačil opcije ob času t:
\[ V_t = premija - min(S_t-K,0) = premija - (S_t-K)^-  \]

Za lažje razumevanje opisanega, si oglejmo spodnjo sliko. Predpostavila
sem naslednje parametre:

\begin{itemize}
\tightlist
\item
  \(K = 20\)
\item
  premija = \(5\)
\end{itemize}

\begin{center}\includegraphics{porocilo_FMF_files/figure-latex/unnamed-chunk-2-1} \end{center}

\begin{enumerate}
\def\labelenumi{\arabic{enumi}.}
\setcounter{enumi}{1}
\tightlist
\item
  PRISTOP K REŠEVANJU PROBLEMA
\end{enumerate}

Reševanja problema se je najbolj smiselno lotiti na način, da vhodne
podatke grafično upodobimo. V nekaterih primerih bo že iz začetne slike
jasno, katero izmed štirih situacij bomo uporabili za aproksimacijo.

Primer `jasne začetne slike' je prikazan na spodnjem grafu.

\begin{center}\includegraphics{porocilo_FMF_files/figure-latex/unnamed-chunk-4-1} \end{center}

Hitro razberemo, da gre za \textbf{nakup evropske put opcije}. Preostane
nam le še določitev parametrov. Določiti moramo izvršilno ceno (\(K\)),
količino (\(Q\)) in premijo (\emph{premium}). Kako to najlažje storimo?

Očitno je, da so vsi štirje tipi opcij sestavljeni iz dveh premic. Ena
izmed premic je vselej vzporedna x osi, druga pa ima bodisi pozitiven
bodisi negativen naklon. Ideja je, da \textbf{vsako vhodno transakcijo
aproksimiramo s kombinacijo teh dveh premic}. S tem, ko določimo
ustrezno kombinacijo premic, lahko hitro ugotovimo, za katero vrsto
opcije in tip pozicije gre. Nadaljno lahko iz smernega koeficienta in
začetne vrednosti izbranih optimalnih premic, določimo iskane parametre.

Premica, ki bo vselej vodoravna določa premijo: \[ y =  premija \]
\emph{Opomba}: premija je lahko negativna ali pozitivna, odvisno katero
pozicijo v opciji zavzamemo.

Premica s pozitivnim ali negativnim naklonom določa količino (Q):

\[ y =  Q * S_t + n\]

Iz presečišča zgornjih dveh premic dobimo izvršilno ceno (K):

\[ K = \frac{premija - n}{Q} \]

\emph{Opomba}: predznak je odvisen od predznaka premije.

Končno iz vsega opisanega sestavimo algoritem, ki bo izračunal iskane
parametre in odgovoril na vprasanje za kateri tip in pozicijo v opciji
gre.

\begin{enumerate}
\def\labelenumi{\arabic{enumi}.}
\setcounter{enumi}{2}
\tightlist
\item
  ALGORITEM
\end{enumerate}

Algoritem sprejme csv datoteko sestavljeno iz dveh stolpcev. V prvem
stolpcu najdemo ceno inštrumenta izrazeno v EUR/MWh in v drugem stolpcu
najdemo izplačilo pri dani ceni, izrazeno v EUR. Algoritem podatke
prebere in najprej določi, za katero vrsto opcije gre. To stori na
naslednji nacin:

\begin{enumerate}
\def\labelenumi{\arabic{enumi}.}
\tightlist
\item
  Algoritem \textbf{izračuna korelacijo} med podatki in s tem izbiro med
  štirimi možnostmi prevede na izbiro med spodnjima dvema:
\end{enumerate}

\begin{itemize}
\tightlist
\item
  če je \textbf{korelacija negativna}, takoj vemo, da gre za \emph{nakup
  put opcije} ali za \emph{prodajo call opcije}.
\end{itemize}

\begin{center}\includegraphics[height=0.5\textheight]{porocilo_FMF_files/figure-latex/unnamed-chunk-5-1} \end{center}

\begin{itemize}
\tightlist
\item
  če je \textbf{korelacija pozitivna}, pa nam preostaneta ali
  \emph{prodaja put opcije} ali pa \emph{nakup call opcije}.
\end{itemize}

\begin{center}\includegraphics[height=0.5\textheight]{porocilo_FMF_files/figure-latex/unnamed-chunk-6-1} \end{center}

\begin{enumerate}
\def\labelenumi{\arabic{enumi}.}
\setcounter{enumi}{1}
\tightlist
\item
  Na drugem koraku izbiramo le še med dvema možnostima. Odločitev ali
  gre za nakupno ali put opcijo sprejmemo na podlagi velikosti napake,
  ki se pojavi pri aproksimaciji z eno ali z drugo kombinacijo premic.
  Izbiramo torej med kombinacijama:
\end{enumerate}

\begin{itemize}
\tightlist
\item
  graf se prične z vodoravno premico in zvezno nadaljuje v premico z
  neničelnim naklonom ali
\item
  graf sestavlja premica z neničelnim naklonom, ki preide v vodoravno
  premico.
\end{itemize}

Na ta način z enostavno primerjavo napak poišemo ustrezno obliko za
aproksimacijo. Prišli smo do jedra algoritma v katerem \textbf{iščemo
optimalno prileganje izbrane opcije na dane podatke}. Potrebno bo iskati
najboljše prileganje obeh omenjenih premic. Premico z neničelnim
naklonom dobimo s pomočjo \textbf{linearne regresije}, medtem ko
vodoravno premico določimo preko \textbf{povprečja profitov} na tistem
delu podatkov, kjer opazimo stacionarnost.

Opisani algoritem nam vrne:

\begin{center}\includegraphics{porocilo_FMF_files/figure-latex/unnamed-chunk-9-1} \end{center}

\begin{verbatim}
## [1] "Gre za nakup put opcije."
## [1] "Približek za izvršilno ceno opcije je 209.369 EUR/MWh, za količino 6744.211 MWh in za premijo 12162.905 EUR."
\end{verbatim}

\begin{center}\includegraphics{porocilo_FMF_files/figure-latex/unnamed-chunk-11-1} \end{center}

\begin{verbatim}
## [1] "Gre za nakup call opcije."
## [1] "Približek za izvršilno ceno opcije je 251.907 EUR/MWh, za količino 5056.285 MWh in za premijo 11160 EUR."
\end{verbatim}

\begin{center}\includegraphics{porocilo_FMF_files/figure-latex/unnamed-chunk-13-1} \end{center}

\begin{verbatim}
## [1] "Gre za prodajo call opcije."
## [1] "Približek za izvršilno ceno opcije je 240.857 EUR/MWh, za količino 5416.354 MWh in za premijo 193440 EUR."
\end{verbatim}

\begin{center}\includegraphics{porocilo_FMF_files/figure-latex/unnamed-chunk-15-1} \end{center}

\begin{verbatim}
## [1] "Gre za nakup call opcije."
## [1] "Približek za izvršilno ceno opcije je 236.36 EUR/MWh, za količino 2376.429 MWh in za premijo 26476.778 EUR."
\end{verbatim}

\begin{enumerate}
\def\labelenumi{\arabic{enumi}.}
\setcounter{enumi}{3}
\tightlist
\item
  KOMENTARJI
\end{enumerate}

Algoritem prepozna tip opcije ter pozicijo, nariše ustrezno
aproksimacijo in vrne iskane parametre. Dobljene aproksimacije se v
nekaterih primerih zdijo precej točne, v drugih nekoliko manj. Na
podatkih kjer imamo več šuma, je aproksimacija (vsaj na videz) nekoliko
slabša, kar je pričakovano. Kar se tiče nadgradnje algoritma, bi bilo
morda smiselno določiti interval zaupanja, torej kolikšna odstopanja
dopuščamo in to upoštevati pri izbiri aproksimacije.

Poleg poročila sem izdelala še shiny aplikacijo, ki sprejme csv
datoteko, izpiše podatke v obliki tabele, nariše graf, aplicira ustrezno
aproksimacijo z evropsko opcijo in vrne iskane parametre. Aplikacija je
dostopna na git repozitoriju.

Za konec, sva razrešila še vprašanje: \emph{Ali se opcijo, kadar je ta
`in the money', vedno splača izvršiti?} Odgovor na vprašanje bi bil
\textbf{ne}. V primerih, ko je tržna cena le malenkost nad \emph{strike
price}, se lahko zgodi, da ob izvršitvi utrpimo izgubo. Razlog za to je
v razliki med ponujeno in zahtevano ceno pri izvedeni transakciji.

\end{document}
