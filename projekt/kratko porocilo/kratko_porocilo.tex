% Options for packages loaded elsewhere
\PassOptionsToPackage{unicode}{hyperref}
\PassOptionsToPackage{hyphens}{url}
%
\documentclass[
]{article}
\title{GEN-I: Nelinearne transakcije}
\usepackage{etoolbox}
\makeatletter
\providecommand{\subtitle}[1]{% add subtitle to \maketitle
  \apptocmd{\@title}{\par {\large #1 \par}}{}{}
}
\makeatother
\subtitle{Kratko porocilo}
\author{Anja Trobec}
\date{April 2022}

\usepackage{amsmath,amssymb}
\usepackage{lmodern}
\usepackage{iftex}
\ifPDFTeX
  \usepackage[T1]{fontenc}
  \usepackage[utf8]{inputenc}
  \usepackage{textcomp} % provide euro and other symbols
\else % if luatex or xetex
  \usepackage{unicode-math}
  \defaultfontfeatures{Scale=MatchLowercase}
  \defaultfontfeatures[\rmfamily]{Ligatures=TeX,Scale=1}
\fi
% Use upquote if available, for straight quotes in verbatim environments
\IfFileExists{upquote.sty}{\usepackage{upquote}}{}
\IfFileExists{microtype.sty}{% use microtype if available
  \usepackage[]{microtype}
  \UseMicrotypeSet[protrusion]{basicmath} % disable protrusion for tt fonts
}{}
\makeatletter
\@ifundefined{KOMAClassName}{% if non-KOMA class
  \IfFileExists{parskip.sty}{%
    \usepackage{parskip}
  }{% else
    \setlength{\parindent}{0pt}
    \setlength{\parskip}{6pt plus 2pt minus 1pt}}
}{% if KOMA class
  \KOMAoptions{parskip=half}}
\makeatother
\usepackage{xcolor}
\IfFileExists{xurl.sty}{\usepackage{xurl}}{} % add URL line breaks if available
\IfFileExists{bookmark.sty}{\usepackage{bookmark}}{\usepackage{hyperref}}
\hypersetup{
  pdftitle={GEN-I: Nelinearne transakcije},
  pdfauthor={Anja Trobec},
  hidelinks,
  pdfcreator={LaTeX via pandoc}}
\urlstyle{same} % disable monospaced font for URLs
\usepackage[margin=1in]{geometry}
\usepackage{color}
\usepackage{fancyvrb}
\newcommand{\VerbBar}{|}
\newcommand{\VERB}{\Verb[commandchars=\\\{\}]}
\DefineVerbatimEnvironment{Highlighting}{Verbatim}{commandchars=\\\{\}}
% Add ',fontsize=\small' for more characters per line
\usepackage{framed}
\definecolor{shadecolor}{RGB}{248,248,248}
\newenvironment{Shaded}{\begin{snugshade}}{\end{snugshade}}
\newcommand{\AlertTok}[1]{\textcolor[rgb]{0.94,0.16,0.16}{#1}}
\newcommand{\AnnotationTok}[1]{\textcolor[rgb]{0.56,0.35,0.01}{\textbf{\textit{#1}}}}
\newcommand{\AttributeTok}[1]{\textcolor[rgb]{0.77,0.63,0.00}{#1}}
\newcommand{\BaseNTok}[1]{\textcolor[rgb]{0.00,0.00,0.81}{#1}}
\newcommand{\BuiltInTok}[1]{#1}
\newcommand{\CharTok}[1]{\textcolor[rgb]{0.31,0.60,0.02}{#1}}
\newcommand{\CommentTok}[1]{\textcolor[rgb]{0.56,0.35,0.01}{\textit{#1}}}
\newcommand{\CommentVarTok}[1]{\textcolor[rgb]{0.56,0.35,0.01}{\textbf{\textit{#1}}}}
\newcommand{\ConstantTok}[1]{\textcolor[rgb]{0.00,0.00,0.00}{#1}}
\newcommand{\ControlFlowTok}[1]{\textcolor[rgb]{0.13,0.29,0.53}{\textbf{#1}}}
\newcommand{\DataTypeTok}[1]{\textcolor[rgb]{0.13,0.29,0.53}{#1}}
\newcommand{\DecValTok}[1]{\textcolor[rgb]{0.00,0.00,0.81}{#1}}
\newcommand{\DocumentationTok}[1]{\textcolor[rgb]{0.56,0.35,0.01}{\textbf{\textit{#1}}}}
\newcommand{\ErrorTok}[1]{\textcolor[rgb]{0.64,0.00,0.00}{\textbf{#1}}}
\newcommand{\ExtensionTok}[1]{#1}
\newcommand{\FloatTok}[1]{\textcolor[rgb]{0.00,0.00,0.81}{#1}}
\newcommand{\FunctionTok}[1]{\textcolor[rgb]{0.00,0.00,0.00}{#1}}
\newcommand{\ImportTok}[1]{#1}
\newcommand{\InformationTok}[1]{\textcolor[rgb]{0.56,0.35,0.01}{\textbf{\textit{#1}}}}
\newcommand{\KeywordTok}[1]{\textcolor[rgb]{0.13,0.29,0.53}{\textbf{#1}}}
\newcommand{\NormalTok}[1]{#1}
\newcommand{\OperatorTok}[1]{\textcolor[rgb]{0.81,0.36,0.00}{\textbf{#1}}}
\newcommand{\OtherTok}[1]{\textcolor[rgb]{0.56,0.35,0.01}{#1}}
\newcommand{\PreprocessorTok}[1]{\textcolor[rgb]{0.56,0.35,0.01}{\textit{#1}}}
\newcommand{\RegionMarkerTok}[1]{#1}
\newcommand{\SpecialCharTok}[1]{\textcolor[rgb]{0.00,0.00,0.00}{#1}}
\newcommand{\SpecialStringTok}[1]{\textcolor[rgb]{0.31,0.60,0.02}{#1}}
\newcommand{\StringTok}[1]{\textcolor[rgb]{0.31,0.60,0.02}{#1}}
\newcommand{\VariableTok}[1]{\textcolor[rgb]{0.00,0.00,0.00}{#1}}
\newcommand{\VerbatimStringTok}[1]{\textcolor[rgb]{0.31,0.60,0.02}{#1}}
\newcommand{\WarningTok}[1]{\textcolor[rgb]{0.56,0.35,0.01}{\textbf{\textit{#1}}}}
\usepackage{graphicx}
\makeatletter
\def\maxwidth{\ifdim\Gin@nat@width>\linewidth\linewidth\else\Gin@nat@width\fi}
\def\maxheight{\ifdim\Gin@nat@height>\textheight\textheight\else\Gin@nat@height\fi}
\makeatother
% Scale images if necessary, so that they will not overflow the page
% margins by default, and it is still possible to overwrite the defaults
% using explicit options in \includegraphics[width, height, ...]{}
\setkeys{Gin}{width=\maxwidth,height=\maxheight,keepaspectratio}
% Set default figure placement to htbp
\makeatletter
\def\fps@figure{htbp}
\makeatother
\setlength{\emergencystretch}{3em} % prevent overfull lines
\providecommand{\tightlist}{%
  \setlength{\itemsep}{0pt}\setlength{\parskip}{0pt}}
\setcounter{secnumdepth}{-\maxdimen} % remove section numbering
\ifLuaTeX
  \usepackage{selnolig}  % disable illegal ligatures
\fi

\begin{document}
\maketitle

NAVODILA ZA IZDELAVO PROJEKTA

Imamo mesecno nelinearno transakcijo za elektricno energijo, kjer se
lahko znotraj dolocenih omejitev za vsako uro znotraj meseca dobave
lastnik opcijskosti odloci, koliko el. energije bo prevzeli/dobavil.
Ovrednotimo jo napram mnozici cenovnih scenarijev, tako da za vsak
cenovni scenarij dobimo njen profit. Cenovni scenariji so mozne
prihodnje cene dobave, oblikovani tako, da ustrezno popisejo
verjetnostno porazdelitev prihodnje cene v smislu njene srednje
vrednosti in standardne deviacije (volatilnosit). Za to transakcijo
zelimo poiskati njen ekvivalent standardne Evropske opcije. Kaksni so
parametri ekvivalenta te opcije, kot kolicina, cena (strike), stran
(nakup/prodaja) in tip (call/put) opcije?

PROJEKT

\begin{enumerate}
\def\labelenumi{\arabic{enumi}.}
\tightlist
\item
  TEORETICNI UVOD
\end{enumerate}

Moja naloga je poiskati Evropsko opcijo, ki najboljse opise transakcijo
predstavljeni v podatkih. Vhodni podatki so pari, v katerih prva
komponenta predstavlja ceno (\emph{\(S_t\)} - \emph{cena ob casu t}) in
druga komponenta profit pri dani ceni (\emph{\(V_t\)} - \emph{vrednost
opcije v casu t}). Cena je izrazena v EUR/MWh, profit pa v EUR.

\[(S_t,V_t) \]

Iskani Evropski opciji, ki bo ekvivalentna transakciji, je potrebno
dolociti najslednje parametre:

\begin{itemize}
\tightlist
\item
  izvrsilno cena (\(K\), \emph{angl. strike price}),
\item
  kolicino (\(Q\)) in
\item
  placano ali prejeto premijo (\emph{angl. option premium}).
\end{itemize}

Dolociti je potrebno tudi \textbf{tip opcije}. Lahko gre za:

\begin{itemize}
\tightlist
\item
  \textbf{nakupno opcijo} (\emph{angl. call option}) ali
\item
  \textbf{prodajno opcijo} (\emph{angl. put option}).
\end{itemize}

Znotraj obeh tipov opcij, locimo se dve poziciji, kateri lahko
zavzamemo. Lahko smo v vlogi izdajatelja opcije (\emph{angl. option
writer}) in v tem primeru \textbf{opcijo prodamo} ali pa \textbf{opcijo
kupimo} in s tem postanemo lastnik opcije (\emph{angl. option buyer}).

To nas pripelje do stirih razlicnih situacij, med katerimi iscemo tisto,
ki najbolje opise dano transakcijo.

OPOMBA: obravnavamo le Evropske opcije.

\begin{center}\includegraphics[height=1.2\textheight]{kratko_porocilo_files/figure-latex/unnamed-chunk-2-1} \end{center}

\begin{enumerate}
\def\labelenumi{\arabic{enumi}.}
\setcounter{enumi}{1}
\tightlist
\item
  PRISTOP K RESEVANJU PROBLEMA
\end{enumerate}

Resevanja problema se je najbolj smiselno lotiti na nacin, da vhodne
podatke graficno upodobimo. V nekaterih primerih bo ze iz zacetne slike
jasno, katero izmed stirih situacij bomo uporabili za aproksimacijo.

Primer `jasne zacetne slike' je prikazan na spodnjem grafu.

\begin{center}\includegraphics{kratko_porocilo_files/figure-latex/unnamed-chunk-4-1} \end{center}

Hitro razberemo, da gre za \textbf{nakup evropske prodajne opcije}.
Preostane nam le se dolocitev parametrov. Dolociti moramo izvrsilno ceno
(\(K\)), kolicino (\(Q\)) in premijo (\emph{premium}). Kako to najlazje
storimo?

Ocitno je, da so vsi stirje tipi opcij sestavljeni iz dveh premic. Ena
izmed premic je vselej vzporedna x osi, druga pa ima bodisi pozitiven
bodisi negativen naklon. Ideja je, da \textbf{vsako vhodno transakcijo
aproksimiramo s kombinacijo teh dveh premic}. S tem, ko dolocimo
ustrezno kombinacijo premic, ugotovimo, za katero vrsto opcije in tip
pozicije gre. Nadaljno lahko iz smernega koeficienta in zacetne
vrednosti izbranih optimalnih premic, dolocimo iskane parametre.

Premica, ki bo vselej vodoravna doloca premijo (\emph{opomba: premija je
lahko negativna ali pozitivna}): \[ y =  premija \]

Premica s pozitivnim ali negativnim naklonom doloca kolicino (Q):

\[ y =  Q * S_t + n\]

Iz presecisca zgornjih dveh premic dobimo izvrsilno ceno (K):

\[ K = \frac{premija - n}{kolicina} \]

Koncno iz vsega opisanega sestavimo algoritem, ki bo izracunal iskane
parametre in odgovoril na vprasanje za kateri tip in pozicijo v opciji
gre.

\begin{enumerate}
\def\labelenumi{\arabic{enumi}.}
\setcounter{enumi}{2}
\tightlist
\item
  ALGORITEM
\end{enumerate}

Algoritem sprejme csv datoteko sestavljeno iz dveh stolpcev. V prvem
stolpcu najdemo ceno financnega instrumenta izrazeno v EUR/MWh in v
drugem stolpcu najdemo profit pri dani ceni, izrazen v EUR. Algoritem
podatke prebere in takoj doloci, za kaksno vrsto opcije gre. To stori na
naslednji nacin:

\begin{enumerate}
\def\labelenumi{\arabic{enumi}.}
\tightlist
\item
  Algoritem \textbf{izracuna korelacijo} med podatki in s tem izbiro med
  stirimi moznostmi prevede na izbiro med spodnjima dvema:
\end{enumerate}

\begin{itemize}
\tightlist
\item
  Ce je \textbf{korelacija negativna}, takoj vemo, da gre za \emph{nakup
  prodajne opcije} ali za \emph{prodajo nakupne opcije}.
\end{itemize}

\begin{center}\includegraphics[height=0.5\textheight]{kratko_porocilo_files/figure-latex/unnamed-chunk-5-1} \end{center}

\begin{itemize}
\tightlist
\item
  Ce je \textbf{korelacija pozitivna}, pa nam preostaneta ali
  \emph{prodaja prodajne opcije} ali pa \emph{nakup nakupne opcije}.
\end{itemize}

\begin{center}\includegraphics[height=0.5\textheight]{kratko_porocilo_files/figure-latex/unnamed-chunk-6-1} \end{center}

\begin{enumerate}
\def\labelenumi{\arabic{enumi}.}
\setcounter{enumi}{1}
\tightlist
\item
  Na drugem koraku izbiramo le se med dvema moznostima. Odlocitev ali
  gre za nakupno ali prodajno opcijo sprejmemo na podlagi velikosti
  napake, ki se pojavi pri aproksimaciji z eno ali z drugo kombinacijo
  premic. Izbiramo torej med kombinacijama:
\end{enumerate}

\begin{itemize}
\tightlist
\item
  prva premica bo vodoravna in druga z nenicelnim naklonom ali
\item
  prva premica bo imela nenicelni naklon in druga bo vodoravna.
\end{itemize}

Z enostavno primerjavo napak smo nasli ustrezno obliko za aproksimacijo.

Prisli smo do jedra algoritma v katerem \textbf{iscemo optimalno
prileganje izbrane opcije na dane podatke}. Potrebno bo iskati najboljse
prileganje obeh omenjenih premic. Premico z nenicelnim naklonom dobimo s
pomocjo \textbf{linearne regresije}, medtem ko vodoravno premico
dolocimo preko \textbf{povprecja profitov} v ``vodoravnem delu
podatkov''.

Opisani algoritem nam vrne:

\begin{center}\includegraphics{kratko_porocilo_files/figure-latex/unnamed-chunk-9-1} \end{center}

\begin{verbatim}
## [1] "Gre za nakup put opcije."
## [1] "Priblizek za izvrsilno ceno opcije je 209.369 EUR/MWh, za kolicino 6744.211 MWh in za premijo 12162.905 EUR."
\end{verbatim}

\begin{enumerate}
\def\labelenumi{\arabic{enumi}.}
\setcounter{enumi}{3}
\tightlist
\item
  UGOTOVITVE IN NADALJNE DELO
\end{enumerate}

Ocitno je, da linearna regresija vsaj na videz ni najboljsa resitev, ko
iscemo optimalni fit. Trenutno delam na izboljsanju aproksimacije
razprsenega dela podatkov. V koncnem porocilu bom predstavila
aproksimacije tudi na drugih podatkih.

Ko bom z aproksimacijo povsem zadovoljna sem razmisljala o vzpostavitvi
shiny aplikacije, ki bi sprejela csv datoteko, narisala graf in vrnila
parametre iskane EU vanilla opcije. O se bolj zanimivih funkcijah
aplikacije bom bolj razmisljala kasneje, ko bo glavni del naloge
opravljen.

\begin{enumerate}
\def\labelenumi{\arabic{enumi}.}
\setcounter{enumi}{4}
\tightlist
\item
  KODA
\end{enumerate}

\begin{Shaded}
\begin{Highlighting}[]
\NormalTok{opt\_fit }\OtherTok{\textless{}{-}} \ControlFlowTok{function}\NormalTok{(price, profit)\{}
  \ControlFlowTok{if}\NormalTok{ (}\FunctionTok{cor}\NormalTok{(price, profit) }\SpecialCharTok{\textless{}} \DecValTok{0}\NormalTok{)\{}
    \CommentTok{\#nakup put opcije (3) ali prodaja nakupne opcije (2)}
    
    \CommentTok{\#pogledamo napake odstopanj in odlocimo ali gre za put ali call}
\NormalTok{    meja }\OtherTok{\textless{}{-}} \FunctionTok{length}\NormalTok{(price)}\SpecialCharTok{/}\DecValTok{4}
\NormalTok{    povpr1 }\OtherTok{\textless{}{-}} \FunctionTok{mean}\NormalTok{(profit[}\DecValTok{1}\SpecialCharTok{:}\NormalTok{meja])}
\NormalTok{    er1 }\OtherTok{\textless{}{-}} \DecValTok{0}
    \ControlFlowTok{for}\NormalTok{ (i }\ControlFlowTok{in} \DecValTok{1}\SpecialCharTok{:}\NormalTok{meja)\{}
\NormalTok{      er1 }\OtherTok{\textless{}{-}}\NormalTok{ er1 }\SpecialCharTok{+}\NormalTok{ (profit[i] }\SpecialCharTok{{-}}\NormalTok{ povpr1)}\SpecialCharTok{\^{}}\DecValTok{2}\NormalTok{\}}
\NormalTok{    povpr1 }\OtherTok{\textless{}{-}} \FunctionTok{mean}\NormalTok{(profit[(}\FunctionTok{length}\NormalTok{(price)}\SpecialCharTok{{-}}\NormalTok{meja)}\SpecialCharTok{:}\FunctionTok{length}\NormalTok{(price)])}
\NormalTok{    er2 }\OtherTok{\textless{}{-}} \DecValTok{0}
    \ControlFlowTok{for}\NormalTok{ (i }\ControlFlowTok{in}\NormalTok{ (}\FunctionTok{length}\NormalTok{(price)}\SpecialCharTok{{-}}\NormalTok{meja)}\SpecialCharTok{:}\FunctionTok{length}\NormalTok{(price))\{}
\NormalTok{      er2 }\OtherTok{\textless{}{-}}\NormalTok{ er2 }\SpecialCharTok{+}\NormalTok{ (profit[i] }\SpecialCharTok{{-}}\NormalTok{ povpr1)}\SpecialCharTok{\^{}}\DecValTok{2}\NormalTok{\}}
    
    
    \CommentTok{\#1. NAKUP PUT OPCIJE\_\_\_\_\_\_\_\_\_\_\_\_\_\_\_\_\_\_\_\_\_\_\_\_\_\_\_\_\_\_\_\_\_\_\_\_\_\_\_\_\_\_\_\_\_\_\_\_\_\_\_\_\_\_\_\_\_\_}
    \ControlFlowTok{if}\NormalTok{ (er1 }\SpecialCharTok{\textgreater{}}\NormalTok{ er2)\{}
\NormalTok{      komentar }\OtherTok{\textless{}{-}} \FunctionTok{paste}\NormalTok{(}\StringTok{"Gre za nakup put opcije."}\NormalTok{)}
      \FunctionTok{print}\NormalTok{(komentar)}
      
\NormalTok{      odstopanje1 }\OtherTok{\textless{}{-}} \FunctionTok{rep}\NormalTok{(}\DecValTok{0}\NormalTok{,}\FunctionTok{length}\NormalTok{(price)) }\CommentTok{\#odstopanje pri aproksimaciji z vodoravno premico}
\NormalTok{      odstopanje2 }\OtherTok{\textless{}{-}} \FunctionTok{rep}\NormalTok{(}\DecValTok{0}\NormalTok{,}\FunctionTok{length}\NormalTok{(price)) }\CommentTok{\#odstopanje pri aproksimaciji z linearno regresijo (posevni del)}
\NormalTok{      najboljsi\_K }\OtherTok{=} \DecValTok{0}
      
      
      \CommentTok{\#GLAVNA ZANKA}
      \ControlFlowTok{for}\NormalTok{ (K }\ControlFlowTok{in} \DecValTok{1}\SpecialCharTok{:}\FunctionTok{length}\NormalTok{(price))\{}
        \CommentTok{\#VODORAVNA PREMICA}
\NormalTok{        premica1 }\OtherTok{\textless{}{-}}\NormalTok{ profit[K]}
\NormalTok{        napaka1 }\OtherTok{\textless{}{-}} \FunctionTok{rep}\NormalTok{(}\DecValTok{0}\NormalTok{,}\FunctionTok{length}\NormalTok{(price[K}\SpecialCharTok{:}\FunctionTok{length}\NormalTok{(profit)]))}
\NormalTok{        profiti }\OtherTok{\textless{}{-}}\NormalTok{ profit[K}\SpecialCharTok{:}\FunctionTok{length}\NormalTok{(profit)]}
        \ControlFlowTok{for}\NormalTok{ (i }\ControlFlowTok{in} \DecValTok{1}\SpecialCharTok{:}\FunctionTok{length}\NormalTok{(napaka1))\{}
\NormalTok{          napaka1[i] }\OtherTok{\textless{}{-}}\NormalTok{ (((premica1 }\SpecialCharTok{{-}}\NormalTok{ profiti[i])}\SpecialCharTok{\^{}}\DecValTok{2}\NormalTok{))}
\NormalTok{        \}}
\NormalTok{        odstopanje1[K] }\OtherTok{\textless{}{-}} \FunctionTok{sum}\NormalTok{(napaka1)}
        
        
        \CommentTok{\#POsEVNA PREMICA}
\NormalTok{        premica2 }\OtherTok{\textless{}{-}} \FunctionTok{lm}\NormalTok{(profit[}\DecValTok{1}\SpecialCharTok{:}\NormalTok{K] }\SpecialCharTok{\textasciitilde{}}\NormalTok{ price[}\DecValTok{1}\SpecialCharTok{:}\NormalTok{K])}
\NormalTok{        odstopanje2[K] }\OtherTok{\textless{}{-}} \FunctionTok{deviance}\NormalTok{(premica2)}
\NormalTok{        odstopanje2[K]}
\NormalTok{      \}}
      
\NormalTok{      odstopanja }\OtherTok{\textless{}{-}}\NormalTok{ odstopanje1 }\SpecialCharTok{+}\NormalTok{ odstopanje2}
      
\NormalTok{      najboljsi\_K }\OtherTok{\textless{}{-}} \FunctionTok{which}\NormalTok{(}\FunctionTok{min}\NormalTok{(odstopanja) }\SpecialCharTok{==}\NormalTok{ odstopanja)}
\NormalTok{      premica1 }\OtherTok{\textless{}{-}} \FunctionTok{mean}\NormalTok{(profit[najboljsi\_K}\SpecialCharTok{:}\FunctionTok{length}\NormalTok{(profit)])}
      \FunctionTok{abline}\NormalTok{(}\AttributeTok{h =}\NormalTok{ profit[najboljsi\_K], }\AttributeTok{col =} \StringTok{\textquotesingle{}red\textquotesingle{}}\NormalTok{, }\AttributeTok{lwd=}\DecValTok{2}\NormalTok{)}
\NormalTok{      premica2 }\OtherTok{\textless{}{-}} \FunctionTok{lm}\NormalTok{(profit[}\DecValTok{1}\SpecialCharTok{:}\NormalTok{najboljsi\_K] }\SpecialCharTok{\textasciitilde{}}\NormalTok{ price[}\DecValTok{1}\SpecialCharTok{:}\NormalTok{najboljsi\_K])}
      \FunctionTok{abline}\NormalTok{(premica2}\SpecialCharTok{$}\NormalTok{coefficients[}\DecValTok{1}\NormalTok{],premica2}\SpecialCharTok{$}\NormalTok{coefficients[}\DecValTok{2}\NormalTok{], }\AttributeTok{col =} \StringTok{\textquotesingle{}dark blue\textquotesingle{}}\NormalTok{,}\AttributeTok{lwd=}\DecValTok{2}\NormalTok{)}
\NormalTok{      najboljsi\_K}
      \CommentTok{\#points(price[najboljsi\_K], profit[najboljsi\_K],type = "p", col = "green")}
      
\NormalTok{      strike\_price }\OtherTok{=} \FunctionTok{round}\NormalTok{(price[najboljsi\_K],}\DecValTok{3}\NormalTok{)}
\NormalTok{      premija }\OtherTok{=} \FunctionTok{round}\NormalTok{(profit[najboljsi\_K],}\DecValTok{3}\NormalTok{)}
\NormalTok{      komentar }\OtherTok{\textless{}{-}} \FunctionTok{paste}\NormalTok{(}\StringTok{"Priblizek za izvrsilno ceno opcije je "}\NormalTok{, }\FunctionTok{as.character}\NormalTok{(strike\_price), }\StringTok{", za premijo pa "}\NormalTok{, }\FunctionTok{as.character}\NormalTok{(premija), }\StringTok{"."}\NormalTok{,}\AttributeTok{sep=}\StringTok{""}\NormalTok{)}
      \FunctionTok{print}\NormalTok{(komentar)}
      
\NormalTok{    \}}
    
    \CommentTok{\#PRODAJA CALL OPCIJE\_\_\_\_\_\_\_\_\_\_\_\_\_\_\_\_\_\_\_\_\_\_\_\_\_\_\_\_\_\_\_\_\_\_\_\_\_\_\_\_\_\_\_\_\_\_\_\_\_\_\_\_\_\_\_\_\_\_}
    
    \ControlFlowTok{if}\NormalTok{ (er2 }\SpecialCharTok{\textgreater{}}\NormalTok{ er1)\{}
\NormalTok{      komentar }\OtherTok{\textless{}{-}} \FunctionTok{paste}\NormalTok{(}\StringTok{"Gre za prodajo call opcije."}\NormalTok{)}
      \FunctionTok{print}\NormalTok{(komentar)}
      
\NormalTok{      odstopanje1 }\OtherTok{\textless{}{-}} \FunctionTok{rep}\NormalTok{(}\DecValTok{0}\NormalTok{,}\FunctionTok{length}\NormalTok{(price)) }\CommentTok{\#odstopanje pri aproksimaciji z vodoravno premico}
\NormalTok{      odstopanje2 }\OtherTok{\textless{}{-}} \FunctionTok{rep}\NormalTok{(}\DecValTok{0}\NormalTok{,}\FunctionTok{length}\NormalTok{(price)) }\CommentTok{\#odstopanje pri aproksimaciji z linearno regresijo (posevni del)}
\NormalTok{      najboljsi\_K }\OtherTok{=} \DecValTok{0}
      
      \ControlFlowTok{for}\NormalTok{ (K }\ControlFlowTok{in} \DecValTok{1}\SpecialCharTok{:}\FunctionTok{length}\NormalTok{(price))\{}
        \CommentTok{\#VODORAVNA PREMICA}
\NormalTok{        premica1 }\OtherTok{\textless{}{-}} \FunctionTok{mean}\NormalTok{(profit[}\DecValTok{1}\SpecialCharTok{:}\NormalTok{K])}
\NormalTok{        napaka1 }\OtherTok{\textless{}{-}} \FunctionTok{rep}\NormalTok{(}\DecValTok{0}\NormalTok{,}\FunctionTok{length}\NormalTok{(price[}\DecValTok{1}\SpecialCharTok{:}\NormalTok{K]))}
        \ControlFlowTok{for}\NormalTok{ (i }\ControlFlowTok{in} \DecValTok{1}\SpecialCharTok{:}\FunctionTok{length}\NormalTok{(napaka1))\{}
\NormalTok{          napaka1[i] }\OtherTok{\textless{}{-}}\NormalTok{ (((premica1 }\SpecialCharTok{{-}}\NormalTok{ profit[i])}\SpecialCharTok{\^{}}\DecValTok{2}\NormalTok{))}
\NormalTok{        \}}
\NormalTok{        odstopanje1[K] }\OtherTok{\textless{}{-}} \FunctionTok{sum}\NormalTok{(napaka1)}
        
        
        \CommentTok{\#POsEVNA PREMICA}
\NormalTok{        premica2 }\OtherTok{\textless{}{-}} \FunctionTok{lm}\NormalTok{(profit[K}\SpecialCharTok{:}\FunctionTok{length}\NormalTok{(price)] }\SpecialCharTok{\textasciitilde{}}\NormalTok{ price[K}\SpecialCharTok{:}\FunctionTok{length}\NormalTok{(price)])}
\NormalTok{        odstopanje2[K] }\OtherTok{\textless{}{-}} \FunctionTok{deviance}\NormalTok{(premica2)}
\NormalTok{        odstopanje2[K]}
\NormalTok{      \}}
      
\NormalTok{      odstopanja }\OtherTok{\textless{}{-}}\NormalTok{ odstopanje1 }\SpecialCharTok{+}\NormalTok{ odstopanje2}
\NormalTok{      najboljsi\_K }\OtherTok{\textless{}{-}} \FunctionTok{which}\NormalTok{(}\FunctionTok{min}\NormalTok{(odstopanja) }\SpecialCharTok{==}\NormalTok{ odstopanja)}
\NormalTok{      premica1 }\OtherTok{\textless{}{-}}\NormalTok{ profit[najboljsi\_K]}
      \FunctionTok{abline}\NormalTok{(}\AttributeTok{h =}\NormalTok{ profit[najboljsi\_K], }\AttributeTok{col =} \StringTok{\textquotesingle{}red\textquotesingle{}}\NormalTok{, }\AttributeTok{lwd=}\DecValTok{2}\NormalTok{)}
\NormalTok{      premica2 }\OtherTok{\textless{}{-}} \FunctionTok{lm}\NormalTok{(profit[najboljsi\_K}\SpecialCharTok{:}\FunctionTok{length}\NormalTok{(price)] }\SpecialCharTok{\textasciitilde{}}\NormalTok{ price[najboljsi\_K}\SpecialCharTok{:}\FunctionTok{length}\NormalTok{(price)])}
      \FunctionTok{abline}\NormalTok{(premica2}\SpecialCharTok{$}\NormalTok{coefficients[}\DecValTok{1}\NormalTok{],premica2}\SpecialCharTok{$}\NormalTok{coefficients[}\DecValTok{2}\NormalTok{], }\AttributeTok{col =} \StringTok{\textquotesingle{}dark blue\textquotesingle{}}\NormalTok{, }\AttributeTok{lwd=}\DecValTok{2}\NormalTok{)}
\NormalTok{      najboljsi\_K}
      \CommentTok{\#points(price[najboljsi\_K], profit[najboljsi\_K],type = "p", col = "green")}
      
\NormalTok{      strike\_price }\OtherTok{=} \FunctionTok{round}\NormalTok{(price[najboljsi\_K],}\DecValTok{3}\NormalTok{)}
\NormalTok{      premija }\OtherTok{=} \FunctionTok{round}\NormalTok{(profit[najboljsi\_K],}\DecValTok{3}\NormalTok{)}
\NormalTok{      komentar }\OtherTok{\textless{}{-}} \FunctionTok{paste}\NormalTok{(}\StringTok{"Priblizek za izvrsilno ceno opcije je "}\NormalTok{, }\FunctionTok{as.character}\NormalTok{(strike\_price), }\StringTok{", za premijo pa "}\NormalTok{, }\FunctionTok{as.character}\NormalTok{(premija), }\StringTok{"."}\NormalTok{,}\AttributeTok{sep=}\StringTok{""}\NormalTok{)}
      \FunctionTok{print}\NormalTok{(komentar)}
\NormalTok{    \}}
\NormalTok{  \}}
  
  \ControlFlowTok{if}\NormalTok{ (}\FunctionTok{cor}\NormalTok{(price, profit) }\SpecialCharTok{\textgreater{}} \DecValTok{0}\NormalTok{)\{}
    \CommentTok{\#nakup call opcije ali prodaja put opcije}
    
\NormalTok{    meja }\OtherTok{\textless{}{-}} \FunctionTok{length}\NormalTok{(price)}\SpecialCharTok{/}\DecValTok{4}
\NormalTok{    povpr1 }\OtherTok{\textless{}{-}} \FunctionTok{mean}\NormalTok{(profit[}\DecValTok{1}\SpecialCharTok{:}\NormalTok{meja])}
\NormalTok{    er1 }\OtherTok{\textless{}{-}} \DecValTok{0}
    
    \ControlFlowTok{for}\NormalTok{ (i }\ControlFlowTok{in} \DecValTok{1}\SpecialCharTok{:}\NormalTok{meja)\{}
\NormalTok{      er1 }\OtherTok{\textless{}{-}}\NormalTok{ er1 }\SpecialCharTok{+}\NormalTok{ (profit[i] }\SpecialCharTok{{-}}\NormalTok{ povpr1)}\SpecialCharTok{\^{}}\DecValTok{2}
\NormalTok{    \}}
    
\NormalTok{    povpr1 }\OtherTok{\textless{}{-}} \FunctionTok{mean}\NormalTok{(profit[(}\FunctionTok{length}\NormalTok{(price)}\SpecialCharTok{{-}}\NormalTok{meja)}\SpecialCharTok{:}\FunctionTok{length}\NormalTok{(price)])}
\NormalTok{    er2 }\OtherTok{\textless{}{-}} \DecValTok{0}
    \ControlFlowTok{for}\NormalTok{ (i }\ControlFlowTok{in}\NormalTok{ (}\FunctionTok{length}\NormalTok{(price)}\SpecialCharTok{{-}}\NormalTok{meja)}\SpecialCharTok{:}\FunctionTok{length}\NormalTok{(price))\{}
\NormalTok{      er2 }\OtherTok{\textless{}{-}}\NormalTok{ er2 }\SpecialCharTok{+}\NormalTok{ (profit[i] }\SpecialCharTok{{-}}\NormalTok{ povpr1)}\SpecialCharTok{\^{}}\DecValTok{2}
\NormalTok{    \}}
    
    \CommentTok{\#PRODAJA PUT OPCIJE\_\_\_\_\_\_\_\_\_\_\_\_\_\_\_\_\_\_\_\_\_\_\_\_\_\_\_\_\_\_\_\_\_\_\_\_\_\_\_\_\_\_\_\_\_\_\_\_\_\_\_\_\_\_\_\_\_\_}
    \ControlFlowTok{if}\NormalTok{ (er1 }\SpecialCharTok{\textgreater{}}\NormalTok{ er2)\{}
\NormalTok{      komentar }\OtherTok{\textless{}{-}} \FunctionTok{paste}\NormalTok{(}\StringTok{"Gre za prodajo put opcije."}\NormalTok{)}
      \FunctionTok{print}\NormalTok{(komentar)}
      
      \CommentTok{\#poiscimo optimalni fit}
\NormalTok{      odstopanje1 }\OtherTok{\textless{}{-}} \FunctionTok{rep}\NormalTok{(}\DecValTok{0}\NormalTok{,}\FunctionTok{length}\NormalTok{(price)) }\CommentTok{\#odstopanje pri aproksimaciji z vodoravno premico}
\NormalTok{      odstopanje2 }\OtherTok{\textless{}{-}} \FunctionTok{rep}\NormalTok{(}\DecValTok{0}\NormalTok{,}\FunctionTok{length}\NormalTok{(price)) }\CommentTok{\#odstopanje pri aproksimaciji z linearno regresijo (posevni del)}
\NormalTok{      najboljsi\_K }\OtherTok{=} \DecValTok{0}
      
      
      \ControlFlowTok{for}\NormalTok{ (K }\ControlFlowTok{in} \DecValTok{1}\SpecialCharTok{:}\FunctionTok{length}\NormalTok{(price))\{}
        \CommentTok{\#VODORAVNA PREMICA}
\NormalTok{        premica1 }\OtherTok{\textless{}{-}} \FunctionTok{mean}\NormalTok{(profit[K}\SpecialCharTok{:}\FunctionTok{length}\NormalTok{(profit)])}
\NormalTok{        napaka1 }\OtherTok{\textless{}{-}} \FunctionTok{rep}\NormalTok{(}\DecValTok{0}\NormalTok{,}\FunctionTok{length}\NormalTok{(price[K}\SpecialCharTok{:}\FunctionTok{length}\NormalTok{(profit)]))}
\NormalTok{        profiti }\OtherTok{\textless{}{-}}\NormalTok{ profit[K}\SpecialCharTok{:}\FunctionTok{length}\NormalTok{(profit)]}
        \ControlFlowTok{for}\NormalTok{ (i }\ControlFlowTok{in} \DecValTok{1}\SpecialCharTok{:}\FunctionTok{length}\NormalTok{(napaka1))\{}
\NormalTok{          napaka1[i] }\OtherTok{\textless{}{-}}\NormalTok{ (((premica1 }\SpecialCharTok{{-}}\NormalTok{ profiti[i])}\SpecialCharTok{\^{}}\DecValTok{2}\NormalTok{))}
\NormalTok{        \}}
\NormalTok{        odstopanje1[K] }\OtherTok{\textless{}{-}} \FunctionTok{sum}\NormalTok{(napaka1)}
        
        
        \CommentTok{\#POsEVNA PREMICA}
\NormalTok{        premica2 }\OtherTok{\textless{}{-}} \FunctionTok{lm}\NormalTok{(profit[}\DecValTok{1}\SpecialCharTok{:}\NormalTok{K] }\SpecialCharTok{\textasciitilde{}}\NormalTok{ price[}\DecValTok{1}\SpecialCharTok{:}\NormalTok{K])}
\NormalTok{        odstopanje2[K] }\OtherTok{\textless{}{-}} \FunctionTok{deviance}\NormalTok{(premica2)}
\NormalTok{        odstopanje2[K]}
\NormalTok{      \}}
      
\NormalTok{      odstopanja }\OtherTok{\textless{}{-}}\NormalTok{ odstopanje1 }\SpecialCharTok{+}\NormalTok{ odstopanje2}
      
\NormalTok{      najboljsi\_K }\OtherTok{\textless{}{-}} \FunctionTok{which}\NormalTok{(}\FunctionTok{min}\NormalTok{(odstopanja) }\SpecialCharTok{==}\NormalTok{ odstopanja)}
\NormalTok{      premica1 }\OtherTok{\textless{}{-}} \FunctionTok{mean}\NormalTok{(profit[najboljsi\_K}\SpecialCharTok{:}\FunctionTok{length}\NormalTok{(profit)])}
      \FunctionTok{abline}\NormalTok{(}\AttributeTok{h =}\NormalTok{ profit[najboljsi\_K], }\AttributeTok{col =} \StringTok{\textquotesingle{}red\textquotesingle{}}\NormalTok{, }\AttributeTok{lwd=}\DecValTok{2}\NormalTok{)}
\NormalTok{      premica2 }\OtherTok{\textless{}{-}} \FunctionTok{lm}\NormalTok{(profit[}\DecValTok{1}\SpecialCharTok{:}\NormalTok{najboljsi\_K] }\SpecialCharTok{\textasciitilde{}}\NormalTok{ price[}\DecValTok{1}\SpecialCharTok{:}\NormalTok{najboljsi\_K])}
      \FunctionTok{abline}\NormalTok{(premica2}\SpecialCharTok{$}\NormalTok{coefficients[}\DecValTok{1}\NormalTok{],premica2}\SpecialCharTok{$}\NormalTok{coefficients[}\DecValTok{2}\NormalTok{], }\AttributeTok{col =} \StringTok{\textquotesingle{}dark blue\textquotesingle{}}\NormalTok{, }\AttributeTok{lwd=}\DecValTok{2}\NormalTok{)}
\NormalTok{      najboljsi\_K}
      \CommentTok{\#points(price[najboljsi\_K], profit[najboljsi\_K],type = "p", col = "green")}
      
\NormalTok{      strike\_price }\OtherTok{=} \FunctionTok{round}\NormalTok{(price[najboljsi\_K],}\DecValTok{3}\NormalTok{)}
\NormalTok{      premija }\OtherTok{=} \FunctionTok{round}\NormalTok{(profit[najboljsi\_K],}\DecValTok{3}\NormalTok{)}
\NormalTok{      komentar }\OtherTok{\textless{}{-}} \FunctionTok{paste}\NormalTok{(}\StringTok{"Priblizek za izvrsilno ceno opcije je "}\NormalTok{, }\FunctionTok{as.character}\NormalTok{(strike\_price), }\StringTok{", za premijo pa "}\NormalTok{, }\FunctionTok{as.character}\NormalTok{(premija), }\StringTok{"."}\NormalTok{,}\AttributeTok{sep=}\StringTok{""}\NormalTok{)}
      \FunctionTok{print}\NormalTok{(komentar)}
      
\NormalTok{    \}}
    
    \CommentTok{\#NAKUP CALL OPCIJE\_\_\_\_\_\_\_\_\_\_\_\_\_\_\_\_\_\_\_\_\_\_\_\_\_\_\_\_\_\_\_\_\_\_\_\_\_\_\_\_\_\_\_\_\_\_\_\_\_\_\_\_\_\_\_\_\_\_}
    \ControlFlowTok{if}\NormalTok{ (er2 }\SpecialCharTok{\textgreater{}}\NormalTok{ er1)\{}
\NormalTok{      komentar }\OtherTok{\textless{}{-}} \FunctionTok{paste}\NormalTok{(}\StringTok{"Gre za nakup call opcije."}\NormalTok{)}
      \FunctionTok{print}\NormalTok{(komentar)}
      
\NormalTok{      odstopanje1 }\OtherTok{\textless{}{-}} \FunctionTok{rep}\NormalTok{(}\DecValTok{0}\NormalTok{,}\FunctionTok{length}\NormalTok{(price)) }\CommentTok{\#odstopanje pri aproksimaciji z vodoravno premico}
\NormalTok{      odstopanje2 }\OtherTok{\textless{}{-}} \FunctionTok{rep}\NormalTok{(}\DecValTok{0}\NormalTok{,}\FunctionTok{length}\NormalTok{(price)) }\CommentTok{\#odstopanje pri aproksimaciji z linearno regresijo (posevni del)}
\NormalTok{      najboljsi\_K }\OtherTok{=} \DecValTok{0}
      
      \ControlFlowTok{for}\NormalTok{ (K }\ControlFlowTok{in} \DecValTok{1}\SpecialCharTok{:}\FunctionTok{length}\NormalTok{(price))\{}
        
        \CommentTok{\#VODORAVNA PREMICA}
\NormalTok{        premica1 }\OtherTok{\textless{}{-}} \FunctionTok{mean}\NormalTok{(profit[}\DecValTok{1}\SpecialCharTok{:}\NormalTok{K])}
\NormalTok{        napaka1 }\OtherTok{\textless{}{-}} \FunctionTok{rep}\NormalTok{(}\DecValTok{0}\NormalTok{,}\FunctionTok{length}\NormalTok{(price[}\DecValTok{1}\SpecialCharTok{:}\NormalTok{K]))}
\NormalTok{        profiti }\OtherTok{\textless{}{-}}\NormalTok{ profit[}\DecValTok{1}\SpecialCharTok{:}\NormalTok{K]}
        \ControlFlowTok{for}\NormalTok{ (i }\ControlFlowTok{in} \DecValTok{1}\SpecialCharTok{:}\FunctionTok{length}\NormalTok{(napaka1))\{}
\NormalTok{          napaka1[i] }\OtherTok{\textless{}{-}}\NormalTok{ (((premica1 }\SpecialCharTok{{-}}\NormalTok{ profiti[i])}\SpecialCharTok{\^{}}\DecValTok{2}\NormalTok{))}
\NormalTok{        \}}
\NormalTok{        odstopanje1[K] }\OtherTok{\textless{}{-}} \FunctionTok{sum}\NormalTok{(napaka1)}
        
        
        \CommentTok{\#POsEVNA PREMICA}
\NormalTok{        premica2 }\OtherTok{\textless{}{-}} \FunctionTok{lm}\NormalTok{(profit[K}\SpecialCharTok{:}\FunctionTok{length}\NormalTok{(price)] }\SpecialCharTok{\textasciitilde{}}\NormalTok{ price[K}\SpecialCharTok{:}\FunctionTok{length}\NormalTok{(price)])}
\NormalTok{        odstopanje2[K] }\OtherTok{\textless{}{-}} \FunctionTok{deviance}\NormalTok{(premica2)}
\NormalTok{        odstopanje2[K]}
        
\NormalTok{      \}}
      
\NormalTok{      odstopanja }\OtherTok{\textless{}{-}}\NormalTok{ odstopanje1 }\SpecialCharTok{+}\NormalTok{ odstopanje2}
\NormalTok{      najboljsi\_K }\OtherTok{\textless{}{-}} \FunctionTok{which}\NormalTok{(}\FunctionTok{min}\NormalTok{(odstopanja) }\SpecialCharTok{==}\NormalTok{ odstopanja)}
\NormalTok{      premica1 }\OtherTok{\textless{}{-}}\NormalTok{ profit[najboljsi\_K]}
      \FunctionTok{abline}\NormalTok{(}\AttributeTok{h =}\NormalTok{ profit[najboljsi\_K], }\AttributeTok{col =} \StringTok{\textquotesingle{}red\textquotesingle{}}\NormalTok{, }\AttributeTok{lwd=}\DecValTok{2}\NormalTok{)}
\NormalTok{      premica2 }\OtherTok{\textless{}{-}} \FunctionTok{lm}\NormalTok{(profit[najboljsi\_K}\SpecialCharTok{:}\FunctionTok{length}\NormalTok{(price)] }\SpecialCharTok{\textasciitilde{}}\NormalTok{ price[najboljsi\_K}\SpecialCharTok{:}\FunctionTok{length}\NormalTok{(price)])}
      \FunctionTok{abline}\NormalTok{(premica2}\SpecialCharTok{$}\NormalTok{coefficients[}\DecValTok{1}\NormalTok{],premica2}\SpecialCharTok{$}\NormalTok{coefficients[}\DecValTok{2}\NormalTok{], }\AttributeTok{col =} \StringTok{\textquotesingle{}dark blue\textquotesingle{}}\NormalTok{, }\AttributeTok{lwd=}\DecValTok{2}\NormalTok{)}
      \CommentTok{\#points(price[najboljsi\_K], profit[najboljsi\_K],type = "p", col = "green")}
      
\NormalTok{      strike\_price }\OtherTok{=} \FunctionTok{round}\NormalTok{(price[najboljsi\_K],}\DecValTok{3}\NormalTok{)}
\NormalTok{      premija }\OtherTok{=} \FunctionTok{round}\NormalTok{(profit[najboljsi\_K],}\DecValTok{3}\NormalTok{)}
\NormalTok{      komentar }\OtherTok{\textless{}{-}} \FunctionTok{paste}\NormalTok{(}\StringTok{"Priblizek za izvrsilno ceno opcije je "}\NormalTok{, }\FunctionTok{as.character}\NormalTok{(strike\_price), }\StringTok{", za premijo pa "}\NormalTok{, }\FunctionTok{as.character}\NormalTok{(premija), }\StringTok{"."}\NormalTok{,}\AttributeTok{sep=}\StringTok{""}\NormalTok{)}
      \FunctionTok{print}\NormalTok{(komentar)}
      
\NormalTok{    \}}
\NormalTok{  \}}
\NormalTok{\}}
\end{Highlighting}
\end{Shaded}


\end{document}
